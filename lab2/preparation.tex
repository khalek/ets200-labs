\documentclass[a4paper]{article}

\usepackage[utf8]{inputenc}

\title{Lab Preparation 2}
\author{}
\date{}

\begin{document}

\maketitle

\section*{Assignment 2}

There are two if statements. The first one will either make the program print ``fail'', in case score is less than 45, or it will print ``pass?'', if score is equal to, or greater than, 45. The second statement will only make the program print `` with distinction'' if score is greater than 80. This produces three different available paths.

To cover all the paths, test cases only has to go into each statement block. To print ``fail'' we only have to have a score less than 45. To print ``pass?'' and `` with distinction'' score has to be at least 45 and at least 80, which only requires one test case. 

\begin{table}[h]
	\begin{tabular}{ll}
		Test Input 	& Expected Output\\
		input(44)	& fail\\
		input(81)	& pass? with distinction\\
	\end{tabular}
\end{table}

\section*{Assignment 3}



\end{document}
